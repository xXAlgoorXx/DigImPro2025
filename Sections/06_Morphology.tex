\section{Morphology}
Mostly used on binary images (can be used on gray level images as well)
\subsection{Binary Morphology}
First threshold image to binary image.
\textbf{Set Operations}:
\[
\begin{array}{rl}
\emptyset & \text{empty set} \\
A \subseteq B & A \text{ is part of } B \\
A \cap B & \text{intersection of } A \text{ and } B \text{: and} \\
A \cup B & \text{union of } A \text{ and } B \text{: or} \\
A^c & \text{complement } A \text{: not} \\
A - B & A \cap B^c : a \in A, a \notin B \\
A_z & A \text{ displaced by } z \text{ pixels: } a \in A : a_x + z_x, a_y + z_y \\
\hat{B} & B \text{ mirrored (point symmetry)}
\end{array}
\]
\subsection{Binary Morphology Operations}
2 basic operations:
\begin{itemize}
    \item Erosion: Eliminates small foreground structures, Shrinks foreground objects
    \[
    f_{erosion}(I) = I \ominus B = \{z \in \mathbb{Z}^2 | B_z \subseteq I\}
    \]
    \item Dilation: Small gaps in foreground structure are filled, Shrinks background objects
    \[
f_{dilation}(I) = I \oplus B = \{z \in \mathbb{Z}^2 | B_z \cap I \neq \emptyset\}
\]
\end{itemize}

Dilation and erosion are dual, i.e. one can be expressed in
terms of the other operation.
\subsection{Compound Operations}
\textbf{Opening}:
\[
f_{open}(I) = (I \ominus B) \oplus B
\]
\begin{itemize}
    \item Erosion followed by a dilation with the same structuring element
    \item Foreground structures smaller than H will be removed (Erosion)
    \item Dilation smooths the remaining structures and let them grow to their original size
\end{itemize}

\textbf{Closing}:
\[
f_{close}(I) = (I \oplus B) \ominus B
\]
\begin{itemize}
    \item Dilation followed by an erosion with the same structuring element
    \item Holes in foreground structures and small gaps smaller than the structuring element are filled
    \item Erosion let the remaining structure grow to its original size
\end{itemize}

\textbf{Boundary Extraction}:
\[
f_{boundary}(I) = I  - (I \ominus B)
\]
Size of B defines the thickness of the border.

\textbf{Hit-or-Miss Transform}:
\[
f_{hit-or-miss}(I)= I \ostar B_{1,2} = (I \ominus B_1) \cap (I^c \ominus B_2)
\]
\begin{itemize}
    \item Basic tool for shape detection and shape extraction
    \item \(B_1\) must be contained in foreground (hit in foreground)
    \item \(B_2\) must be contained in background (hit in background, miss in foreground)
\end{itemize}

\subsection{Applications}
\begin{itemize}
    \item Convex Hull
    \item Thinning
    \item Skeletons
    \item Pruning
\end{itemize}

\subsection{Grayscale Morphology}
2 kind of structuring elements:
\begin{itemize}
    \item Flat: All pixels have the same weight
    \item Non-flat: Pixels have different weights
\end{itemize}
\textbf{Grayscale Erosion}:
\[
[I \ominus b](x, y) = \min_{(s, t) \in b} \left\{ I(x + s, y + t) - b(s, t) \right\}
\]

\textbf{Grayscale Dilation}:
\[
[I \oplus b](x, y) = \max_{(s, t) \in \hat{b}} \left\{ I(x - s, y - t) + \hat{b}(s, t) \right\}
\]

\textbf{Opening / Closing (with flat structuring element)}

Opening: supresses bright details

Closing: supresses dark details

\subsection{Top Hat/Bottom Hat - Transformation}
\textbf{Top Hat}:
\[
f_{top-hat}(I) = I - (I \ominus B)
\]
Removes light objects on a dark background.

\textbf{Bottom Hat}:
\[
f_{bottom-hat}(I) = (I \oplus B) - I
\]
Removes dark objects on a bright
background