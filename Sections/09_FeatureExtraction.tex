\section{Feature, Feature Extraction and Compersion}
We destinguish:
\begin{itemize}
    \item \textbf{Key points:} Image coordinates of a destinctive region of an image
    \item \textbf{Feature Descriptor:} A numeric vector that describes the image region around the key point
\end{itemize}
\textbf{Feature Detection:} Finding features (e.g. corners) in the image

\textbf{Description:} Assigning quantitative/qualitative attributes to the features(e.g. corner orientation, relative position)


Desirable Feature Characteristics:
\begin{itemize}
    \item \textbf{Independence} from location, rotaion and scale
    \item \textbf{Robustness} to changes in illumination and viewpoint
    \item \textbf{Preprocessing} can help normalise images(e.g. via histogram equalization) to improve feature detection and description
\end{itemize}

% \subsection{Object detection \& Recognition}
% \begin{enumerate}
%     \item Edges
%     \item corners
%     \item Local Description
%     \item Shape Features
% \end{enumerate}
\subsection{Features and location}
Types of image features:
\begin{itemize}
    \item \textbf{Corners:} High curvature in multiple directions
    \item \textbf{Edges:} High gradient in one direction
    \item \textbf{Blobs/Regions:} Areas differ in intensitiy/colour from sourroundings
    \item \textbf{Descriptors:} Numerical vectors that encode local appearance
\end{itemize}

\subsubsection{Harris Corner Detection}
\[
I_x = I * \begin{bmatrix}
-1 & 0 & 1 \\
-2 & 0 & 2 \\
-1 & 0 & 1
\end{bmatrix}, \quad
I_y = I * \begin{bmatrix}
-1 & -2 & -1 \\
0 & 0 & 0 \\
1 & 2 & 1
\end{bmatrix}
\]
\[
M = \begin{bmatrix}
\sum w \cdot I_x^2 & \sum w \cdot I_x I_y \\
\sum w \cdot I_x I_y & \sum w \cdot I_y^2
\end{bmatrix}, \quad
M = \begin{bmatrix}
A & C \\
C & B
\end{bmatrix}
\]
\[
R = \det(M) - k \cdot (\operatorname{trace}(M))^2 = (AB - C^2) - k(A + B)^2
\]
\begin{itemize}
    \item \(R > 0:\) Corner region
    \item \(R < 0:\) Edge region
    \item \(|R|\) small: Flat region
\end{itemize}

\subsubsection{SIFT (Scale-Invariant Feature Transform)}
Finds key points invariant to scale, rotation and limited viewpoint changes.

Algorithm:
\begin{enumerate}
    \item Build Gaussian scale space pyramid
    \item Compute the difference of Gaussian (DoG) pyramid
    \item Detect local extrema in DoG scale space
    \item Perform key point localisation and filterning 
    \item Assign orientation to key points
    \item Generate 128-dim SIFT descriptor
\end{enumerate}

\subsection{SURF (Speeded-Up Robust Features)}
Scale invariant, Rotation invariant and parial illumination invariant feature detector and descriptor.
\textbf{Integral Image:} Each pixel(\(x,y\)) contains the sum of all pixels above and to the left and above, allowing fast rectangular region sums.

Algorithm:
\begin{enumerate}
    \item Compute integral image
    \item Detect interest points using Hessian matrix
    \item Assign orientation using Haar wavelet responses
    \item Construct SURF descriptor vectors
\end{enumerate}

\subsubsection{BRIEF (Binary Robust Invariant Elementary Features)}
Generates binary strings. Extermely fast, Compact (binary String) and effiecient for matching via Hamming distance.
\begin{enumerate}
    \item \textbf{Keypoitn Detector:} BRIEF assumes you already have a set of key points (e.g. from FAST,ORB or any corner detector)
    \item \textbf{Local Path:} Around each key point define a small patch (e.g. 31x31 pixels)
    \item \textbf{Pairwise Intensity Test:} Within this patch, BRIEF randomly samples pairs of points (\(p_i,q_i\)) and compares their intensities
    \item \textbf{Binary Bit String:} For each pair, \(I(p_i)<I(q_i)\) then output '1', else '0'. Concatenate all comparisons to form the final binary descriptor.
    \[
    bit_i = 
    \begin{cases}
    1 & \text{if } I(p_i) < I(q_i) \\
    0 & \text{otherwise}
    \end{cases}
    \]
    \item \textbf{Lenght of Descriptor:} The number of (\(p,q\)) comperisons. For example BRIEF-128 uses 128 comperisons per patch, resulting in a 128-bit descriptor
\end{enumerate}
\textbf{Limitations:}
\begin{itemize}
    \item No Built-in Rotation or Scale Invariance
    \item Sensitivity to Noise \& Blur
    \item Image Boundary Issues
\end{itemize}

\subsubsection{FAST(Feature from Accelerated Segment Test)}
Fast is not a full feature descriptor, its \textbf{primarily keypoint} detector(i.e. where corners exist in image). Often combined with a seperate descriptor.
\textbf{High Speed}(near realtime),\textbf{High repeatability}.

\subsubsection{ORB (Oriented FAST and Rotated BRIEF)}
\begin{itemize}
    \item \textbf{Speed:} order of magnitude faster than SIFT
    \item \textbf{Rotation Invariance:} Through orientation assignment and rotating BRIEF, ORB is robust to inplane rotations
    \item \textbf{Partial Scale Handeling:} By detection FAST corners in a pyramid, ORB can capture features at multiple scales. However, its not as explicity scale-invariant as SIFT's Gaussian scale space
\end{itemize}
Practical Application:
\begin{itemize}
    \item SLAM in robotics
    \item Mobile AR
    \item Real-time Tracking
\end{itemize}
NMS(Non-Maximum Suppresion) as post-processing. Goal:
\begin{itemize}
    \item retain only the most significant 'response'
    \item eliminate weaker, nearby responses
\end{itemize}
\subsection{Matching Features}
\subsubsection{Angle comparison}
Angle comparison for float feature vectors. Let \(d_1^n,d_2^n\) be a n-dimensional real-number feature vectors:
\[
\cos(\alpha) = \frac{a\cdot b}{|a|\cdot|b|}
\]
Similar features will have small angle,\(\cos(\alpha) \approx 1\).
\subsubsection{Hamming Distance}
Hamming distance \(D_H(\mathbf{a},\mathbf{b}) \ge 0\) Measures dissimilarity between two binary vectors. Non negative, symmetric.

Image A Hash : 1011 0011

Image B Hash : 1001 1011

Differences : Pos.2 \quad $0 \neq 0$ ,Pos.6 \quad $0 \neq 1$

\(D_H(A,B) = 2\)


